%%%%%%%%%%%%%%%%%%%%%%%%%%%%%%%%%%%%%%%%%%%%%%%%%%%%%%%%%%%%%%%%%%%%%%%%
%%%%%%%%%%%%%%%%%%%%%% Simple LaTeX CV Template %%%%%%%%%%%%%%%%%%%%%%%%
%%%%%%%%%%%%%%%%%%%%%%%%%%%%%%%%%%%%%%%%%%%%%%%%%%%%%%%%%%%%%%%%%%%%%%%%

%%%%%%%%%%%%%%%%%%%%%%%%%%%%%%%%%%%%%%%%%%%%%%%%%%%%%%%%%%%%%%%%%%%%%%%%
%% NOTE: If you find that it says                                     %%
%%                                                                    %%
%%                           1 of ??                                  %%
%%                                                                    %%
%% at the bottom of your first page, this means that the AUX file     %%
%% was not available when you ran LaTeX on this source. Simply RERUN  %%
%% LaTeX to get the ``??'' replaced with the number of the last page  %%
%% of the document. The AUX file will be generated on the first run   %%
%% of LaTeX and used on the second run to fill in all of the          %%
%% references.                                                        %%
%%%%%%%%%%%%%%%%%%%%%%%%%%%%%%%%%%%%%%%%%%%%%%%%%%%%%%%%%%%%%%%%%%%%%%%%

%%%%%%%%%%%%%%%%%%%%%%%%%%%% Document Setup %%%%%%%%%%%%%%%%%%%%%%%%%%%%

% Don't like 10pt? Try 11pt or 12pt
\documentclass[9.5pt]{article}

% This is a helpful package that puts math inside length specifications
\usepackage{calc}

%\usepackage{enumitem}


% Simpler bibsection for CV sections
% (thanks to natbib for inspiration)
\makeatletter
\newlength{\bibhang}
\setlength{\bibhang}{1em}
\newlength{\bibsep}
 {\@listi \global\bibsep\itemsep \global\advance\bibsep by\parsep}
\newenvironment{bibsection}%
        {\vspace{-\baselineskip}\begin{list}{}{%
       \setlength{\leftmargin}{\bibhang}%
       \setlength{\itemindent}{-\leftmargin}%
       \setlength{\itemsep}{\bibsep}%
       \setlength{\parsep}{\z@}%
        \setlength{\partopsep}{0pt}%
        \setlength{\topsep}{0pt}}}
        {\end{list}\vspace{-.6\baselineskip}}
\makeatother

% Layout: Puts the section titles on left side of page
\reversemarginpar

%
%         PAPER SIZE, PAGE NUMBER, AND DOCUMENT LAYOUT NOTES:
%
% The next \usepackage line changes the layout for CV style section
% headings as marginal notes. It also sets up the paper size as either
% letter or A4. By default, letter was used. If A4 paper is desired,
% comment out the letterpaper lines and uncomment the a4paper lines.
%
% As you can see, the margin widths and section title widths can be
% easily adjusted.
%
% ALSO: Notice that the includefoot option can be commented OUT in order
% to put the PAGE NUMBER *IN* the bottom margin. This will make the
% effective text area larger.
%
% IF YOU WISH TO REMOVE THE ``of LASTPAGE'' next to each page number,
% see the note about the +LP and -LP lines below. Comment out the +LP
% and uncomment the -LP.
%
% IF YOU WISH TO REMOVE PAGE NUMBERS, be sure that the includefoot line
% is uncommented and ALSO uncomment the \pagestyle{empty} a few lines
% below.
%

%% Use these lines for letter-sized paper
%\usepackage[paper=letterpaper,
%            %includefoot, % Uncomment to put page number above margin
%            marginparwidth=1.2in,     % Length of section titles
%           marginparsep=.05in,       % Space between titles and text
%            margin=1in,               % 1 inch margins
%            includemp]{geometry}

% Use these lines for A4-sized paper
\usepackage[paper=a4paper,
            %includefoot, % Uncomment to put page number above margin
            marginparwidth=25mm,    % Length of section titles
            marginparsep=5mm,       % Space between titles and text
            margin=20mm,              % 25mm margins
            includemp]{geometry}

%% More layout: Get rid of indenting throughout entire document
\setlength{\parindent}{0in}

%% This gives us fun enumeration environments. compactitem will be nice.
\usepackage{paralist}

%% Reference the last page in the page number
%
% NOTE: comment the +LP line and uncomment the -LP line to have page
%       numbers without the ``of ##'' last page reference)
%
% NOTE: uncomment the \pagestyle{empty} line to get rid of all page
%       numbers (make sure includefoot is commented out above)
%
\usepackage{fancyhdr,lastpage}
\pagestyle{fancy}
%\pagestyle{empty}      % Uncomment this to get rid of page numbers
\fancyhf{}\renewcommand{\headrulewidth}{0pt}
\fancyfootoffset{\marginparsep+\marginparwidth}
\newlength{\footpageshift}
\setlength{\footpageshift}
          {0.5\textwidth+0.5\marginparsep+0.5\marginparwidth-2in}
\lfoot{\hspace{\footpageshift}%
       \parbox{4in}{\, \hfill %
                    \arabic{page} of \protect\pageref*{LastPage} % +LP
%                    \arabic{page}                               % -LP
                    \hfill \,}}

% Finally, give us PDF bookmarks
\usepackage{color,hyperref}
\definecolor{darkblue}{rgb}{0.0,0.0,0.3}
\hypersetup{colorlinks,breaklinks,
            linkcolor=darkblue,urlcolor=darkblue,
            anchorcolor=darkblue,citecolor=darkblue}

%%%%%%%%%%%%%%%%%%%%%%%% End Document Setup %%%%%%%%%%%%%%%%%%%%%%%%%%%%


%%%%%%%%%%%%%%%%%%%%%%%%%%% Helper Commands %%%%%%%%%%%%%%%%%%%%%%%%%%%%

% The title (name) with a horizontal rule under it
% (optional argument typesets an object right-justified across from name
%  as well)
%
% Usage: \makeheading{name}
%        OR
%        \makeheading[right_object]{name}
%
% Place at top of document. It should be the first thing.
% If ``right_object'' is provided in the square-braced optional
% argument, it will be right justified on the same line as ``name'' at
% the top of the CV. For example:
%
%       \makeheading[\emph{Curriculum vitae}]{Your Name}
%
% will put an emphasized ``Curriculum vitae'' at the top of the document
% as a title. Likewise, a picture could be included:
%
%   \makeheading[\includegraphics[height=1.5in]{my_picutre}]{Your Name}
%
% the picture will be flush right across from the name.
\newcommand{\makeheading}[2][]%
        {\hspace*{-\marginparsep minus \marginparwidth}%
         \begin{minipage}[t]{\textwidth+\marginparwidth+\marginparsep}%
             {\large \bfseries #2 \hfill #1}\\[-0.15\baselineskip]%
                 \rule{\columnwidth}{1pt}%
         \end{minipage}}

% The section headings
%
% Usage: \section{section name}
%
% Follow this section IMMEDIATELY with the first line of the section
% text. Do not put whitespace in between. That is, do this:
%
%       \section{My Information}
%       Here is my information.
%
% and NOT this:
%
%       \section{My Information}
%
%       Here is my information.
%
% Otherwise the top of the section header will not line up with the top
% of the section. Of course, using a single comment character (%) on
% empty lines allows for the function of the first example with the
% readability of the second example.
\renewcommand{\section}[2]%
        {\pagebreak[3]\vspace{1.3\baselineskip}%
         \phantomsection\addcontentsline{toc}{section}{#1}%
         \hspace{0in}%
         \marginpar{
         \raggedright \scshape #1}#2}

% An itemize-style list with lots of space between items
\newenvironment{outerlist}[1][\enskip\textbullet]%
        {\begin{itemize}[#1]}{\end{itemize}%
         \vspace{-.6\baselineskip}}

% An environment IDENTICAL to outerlist that has better pre-list spacing
% when used as the first thing in a \section
\newenvironment{lonelist}[1][\enskip\textbullet]%
        {\vspace{-\baselineskip}\begin{list}{#1}{%
        \setlength{\partopsep}{0pt}%
        \setlength{\topsep}{0pt}}}
        {\end{list}\vspace{-.6\baselineskip}}

% An itemize-style list with little space between items
\newenvironment{innerlist}[1][\enskip\textbullet]%
        {\begin{compactitem}[#1]}{\end{compactitem}}

% An environment IDENTICAL to innerlist that has better pre-list spacing
% when used as the first thing in a \section
\newenvironment{loneinnerlist}[1][\enskip\textbullet]%
        {\vspace{-\baselineskip}\begin{compactitem}[#1]}
        {\end{compactitem}\vspace{-.6\baselineskip}}

% To add some paragraph space between lines.
% This also tells LaTeX to preferably break a page on one of these gaps
% if there is a needed pagebreak nearby.
\newcommand{\blankline}{\quad\pagebreak[3]}
\newcommand{\halfblankline}{\quad\vspace{-0.5\baselineskip}\pagebreak[3]}

% Uses hyperref to link DOI
\newcommand\doilink[1]{\href{http://dx.doi.org/#1}{#1}}
\newcommand\doi[1]{doi:\doilink{#1}}

% For \url{SOME_URL}, links SOME_URL to the url SOME_URL
\providecommand*\url[1]{\href{#1}{#1}}
% Same as above, but pretty-prints SOME_URL in teletype fixed-width font
\renewcommand*\url[1]{\href{#1}{\texttt{#1}}}

% For \email{ADDRESS}, links ADDRESS to the url mailto:ADDRESS
\providecommand*\email[1]{\href{mailto:#1}{#1}}
% Same as above, but pretty-prints ADDRESS in teletype fixed-width font
%\renewcommand*\email[1]{\href{mailto:#1}{\texttt{#1}}}

%\providecommand\BibTeX{{\rm B\kern-.05em{\sc i\kern-.025em b}\kern-.08em
%    T\kern-.1667em\lower.7ex\hbox{E}\kern-.125emX}}
%\providecommand\BibTeX{{\rm B\kern-.05em{\sc i\kern-.025em b}\kern-.08em
%    \TeX}}
\providecommand\BibTeX{{B\kern-.05em{\sc i\kern-.025em b}\kern-.08em
    \TeX}}
\providecommand\Matlab{\textsc{Matlab}}

%%%%%%%%%%%%%%%%%%%%%%%% End Helper Commands %%%%%%%%%%%%%%%%%%%%%%%%%%%

%%%%%%%%%%%%%%%%%%%%%%%%% Begin CV Document %%%%%%%%%%%%%%%%%%%%%%%%%%%%

\begin{document}
\makeheading{Jiajie Peng}

\section{Contact Information}
%
% NOTE: Mind where the & separators and \\ breaks are in the following
%       table.
%
% ALSO: \rcollength is the width of the right column of the table
%       (adjust it to your liking; default is 1.85in).
%
%\newlength{\rcollength}\setlength{\rcollength}{2.05in}%
%
\begin{tabular}[t]{ll}
School of Computer Science and Technolodgy& \textit{Voice:} +86-187 4513 7124 \\
315 Zonghe Building, 92 West Dazhi Street , & \textit{E-mail:} \email{jiajiepeng@hit.edu.cn}\\
Nan Gang District, Harbin, China  & %\textit{Homepage:} \href{http://www.hlt.utdallas.edu/~zywei}{http://www.hlt.utdallas.edu/$\scriptsize{\sim}$zywei}\\
\end{tabular}

%\section{Objective}
%
%Placement in an academic position (i.e., faculty, postdoctoral, or
%research scientist) that allows for advanced research in distributed
%complex adaptive systems (i.e., modeling, analysis, design, and
%verification) with a particular focus on the control of engineered
%agents (e.g., for communications, control, software, and electronics
%applications) and the analysis of biological phenomena (e.g.,
%self-organization, ecological rationality)
%
\section{Research Interests}
%
Bioinformatics, data mining, data-driven ontology construction, gene ontology based semantic similarity, biological network reconstruction 

%\section{Current Position}
%\textbf{University of Texas at Dallas (UTD)}, Richardson, TX, USA \hfill Dec. 2014 - Present \\
%Postdoctoral Associate, Human Language Technology Research Institute\\
%Mentor: Yang Liu

\section{Education}
\textbf{Harbin Institute of Technolodgy (HIT)}, Harbin, China \hfill Aug. 2010 - now \\
Ph.D., Computer Science and Technolodgy\\
Advisor: Yadong Wang\\
\\
\textbf{Michigan State University (MSU)}, East Lansing, MI, USA\\
Visiting Scholar, MSU-DOE Plant Research Laboratory \hfill Sep. 2010 - Sep. 2012\\
Mentor: Jin Chen\\

\textbf{Harbin Institute of Technology (HIT)}, Harbin, China\\
M.Phil., Computer Science and Technology \hfill Aug. 2008 - Jul. 2010\\
B.Sc., Computer Science and Technology \hfill Aug. 2004 - Jul. 2008\\


\section{Publications}
%\textbf{- Conference Papers}

\begin{itemize}
	\item \textbf{Peng J}, Li H, Wang Y and Chen J, \emph{A web tool for measuring gene semantic similarities by combining Gene Ontology and gene co-function networks}, proceedings of the 6th ACM Conference on Bioinformatics, Computational Biology (ACM BCB15), 2015, in press.
%
	\item \textbf{Peng J}, Li H, Liu Y, Juan L, Jiang Q, Wang Y and Chen J, \emph{InteGO2: a Web Tool for Measuring and Visualizing Gene Semantic Similarities using Gene Ontology}, BMC Genomics, 2015, in press
	%
	\item \textbf{Peng J}, Wang T, Hu J, Wang Y and Chen J, \emph{Constructing organelle association network in Arabidopsis thaliana}, Current Genomics, 2015, in press.

%
	\item Liu Y, Liu J, Lu J, \textbf{Peng J}, Juan L, Zhu X, Li B and Wang Y, \emph{Joint detection of copy number variations in parent-offspring trios}, Bioinformatics, 2015, in press.
		%Peng J, Uygun S, Kim T, Wang Y*, Rhee SY* and Chen J*. Measuring semantic similarities by combining gene ontology annotations and gene co-function networks. BMC Bioinformatics, 2015, 16:44.
	\item \textbf{Peng J}, Uygun S, Kim T, Wang Y, Rhee SY and Chen J, \emph{Measuring semantic similarities by combining gene ontology annotations and gene co-function networks}, BMC Bioinformatics, 2015, 16:44.
%
	\item Jiang Q, Ma R, Wang J, Wu X, Jin S, \textbf{Peng J}, Tan R, Zhang T, Li Y and  Wang Y, \emph{LncRNA2Function: a comprehensive resource for functional investigation of human lncRNAs based on RNA-seq data}, BMC genomics, 2015, 16:S2.
%
	\item Jiang Q, Wang J, Wu X, Ma R, Jin S, Han Z,  Tan R, \textbf{Peng J}, Liu G, Li Y and  Wang Y, \emph{LncRNA2Target: a database for differentially expressed genes after lncRNA knockdown or overexpression}, Nucleic Acids Res, 2015, 43:D193-D196.
%
	\item \textbf{Peng J}, Wang Y, and Chen J, \emph{Towards integrative gene functional similarity measurement}, BMC Bioinformatics, 2014, 15:S5.
%
	\item Cheng L, Li J, Ju P, \textbf{Peng J} and Wang Y, \emph{SemFunSim: A New Method for Measuring Disease Similarity by Integrating Semantic and Gene Functional Association}, PLoS ONE, 2014, 9:e99415.
%
	\item \textbf{Peng J}, Li H, Jiang Q, Wang Y and Chen J, \emph{An Integrative Approach for Measuring Semantic Similarities using Gene Ontology}, BMC Systems Biology, 2014, 8:S8.
%
	\item Jin S, Tan R, Jiang Q, Xu L, \textbf{Peng J}, Wang Yo and Wang Y, \emph{A  Generalized Topological Entropy for Analyzing the Complexity of DNA Sequences}, PLoS ONE, 2014, 9:e88519 .
%
	\item \textbf{Peng J}, Chen J and Wang Y, \emph{Identifying cross-category relations in gene ontology and constructing genome-specific term association networks}, BMC Bioinformatics, 2013, 14:S15.
%
	\item \textbf{Peng J}, Wang T,  Wang Y and Chen J, \emph{Extending Gene Ontology with Gene Network Data}, Bioinformatics, Major Revision.
\end{itemize}


\section{Research}
%\textbf{University of Texas at Dallas}\\
%\emph{Postdoctoral Associate} \hfill Dec. 2014 - Present\\

\section{Professional Skills}
\textbf{General}: Data mining for Bioinformatics, multi-omics data analysis, fast modeling and prototyping, familiar with network construction and analysis, familiar with ontology based data analysis% data analysis, familiar with human annotation collection for NLP task%, familiar with Twitter data processing

\textbf{Programming}: Java, C/C++, R, Unix/Linux shell scripting, Python


\section{Teaching Experience}
\textbf{Teaching Assistant} at School of Computer Science and Technolodgy, HIT\\

Graduate course "Knowledge Engineering" \hfill Spring 2014, 2015\\
Undergraduate course "Introduction to bioinformatics" \hfill Spring 2014\\
%Undergraduate course "Information Technology Management" \hfill Fall 2012, 2013\\
%Undergraduate course "Information Systems Management" \hfill Fall 2011\\

\section{Mentorship}
Qinghua Jiang, Associate Professor, Harbin Institute of Technolodgy \hfill Spring 2013 - Present\\
Liang Cheng, Assistant Professor, Harbin Medical University \hfill Spring 2013 - Present\\
Yongzhuang Liu, PhD, Harbin Institute of Technolodgy \hfill Spring 2014 - Present\\



\section{Honors and Awards}
China Scholarship Council Scholarship for Postgraduate Student\hfill 2010\\
National Scholarship for Postgraduate Student (top 1\%) \hfill 2013 \\
Guanghua Scholarship for Postgraduate Student (top 1\%) \hfill 2014 
%Excellent Graduate in Heilongjiang Province (top 1\%) \hfill 2010\\
%Excellent Postgraduate Student in HIT (top 1\%) \hfill 2010\\
%Excellent Graduate in Heilongjiang Province (top 1\%) \hfill 2008 \\
%Excellent Student Leader in Heilongjiang Province (top 1\%) \hfill 2006\\

\section{Referees}

\begin{tabular}{ll}
{\textbf{Yadong Wang}} & \href{https://www.msu.edu/~jinchen/}{\textbf{Jin Chen}}\\
Professor & PhD, Assistant Professor\\
Harbin Institute of Technolodgy & Michigan State University\\
Email: ydwang@hit.edu.cn & Email: jinchen@msu.edu\\
\\
%\href{http://www1.aston.ac.uk/eas/staff/dr-yulan-he/} {\textbf{Dr. Yulan HE}} & \href{http://cs.hitsz.edu.cn/teachers/t1/1181827169.html}{\textbf{Professor Xiaolong} WANG}\\
%PhD, Senior lecturer & PhD, Professor\\
%Aston University & Harbin Institute of Technology\\
%Email: y.he9@aston.ac.uk & Email: wangxl@insun.hit.edu.cn\\
%\end{tabular}

%\href{http://www.cintec.cuhk.edu.hk/kfwong/}{\textbf{Professor Kam-Fai WONG}} & \href{http://www1.se.cuhk.edu.hk/~hcheng/}{\textbf{Prof. Hong CHENG}}\\
%PhD, Professor & PhD, Assistant Professor\\
%The Chinese University of Hong Kong & The Chinese University of Hong Kong\\
%Email: kfwong@se.cuhk.edu.hk & Email: hcheng@se.cuhk.edu.hk\\
%\\
\href{http://genome.wustl.edu/people/individual/bruce-rosa/} {\textbf{Bruce A Rosa}} & \\
PhD, Staff scientist & \\
Washington University in St. Louis & \\
Email: barosa@lakeheadu.ca &\\

%\href{http://www1.aston.ac.uk/eas/staff/dr-yulan-he/} {\textbf{Yulan He}} &\href{http://www.hlt.utdallas.edu/~yangl}{\textbf{Yang Liu}} \\
%PhD, Senior lecturer & PhD, Associate Processor \\
%Aston University & The University of Texas at Dallas\\
%Email: y.he9@aston.ac.uk &Email: yangl@hlt.utdallas.edu\\
\end{tabular}

\end{document} 