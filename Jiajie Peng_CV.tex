%%%%%%%%%%%%%%%%%%%%%%%%%%%%%%%%%%%%%%%%%%%%%%%%%%%%%%%%%%%%%%%%%%%%%%%%
%%%%%%%%%%%%%%%%%%%%%% Simple LaTeX CV Template %%%%%%%%%%%%%%%%%%%%%%%%
%%%%%%%%%%%%%%%%%%%%%%%%%%%%%%%%%%%%%%%%%%%%%%%%%%%%%%%%%%%%%%%%%%%%%%%%

%%%%%%%%%%%%%%%%%%%%%%%%%%%%%%%%%%%%%%%%%%%%%%%%%%%%%%%%%%%%%%%%%%%%%%%%
%% NOTE: If you find that it says                                     %%
%%                                                                    %%
%%                           1 of ??                                  %%
%%                                                                    %%
%% at the bottom of your first page, this means that the AUX file     %%
%% was not available when you ran LaTeX on this source. Simply RERUN  %%
%% LaTeX to get the ``??'' replaced with the number of the last page  %%
%% of the document. The AUX file will be generated on the first run   %%
%% of LaTeX and used on the second run to fill in all of the          %%
%% references.                                                        %%
%%%%%%%%%%%%%%%%%%%%%%%%%%%%%%%%%%%%%%%%%%%%%%%%%%%%%%%%%%%%%%%%%%%%%%%%

%%%%%%%%%%%%%%%%%%%%%%%%%%%% Document Setup %%%%%%%%%%%%%%%%%%%%%%%%%%%%

% Don't like 10pt? Try 11pt or 12pt
\documentclass[9.5pt]{article}

% This is a helpful package that puts math inside length specifications
\usepackage{calc}

%\usepackage{enumitem}


% Simpler bibsection for CV sections
% (thanks to natbib for inspiration)
\makeatletter
\newlength{\bibhang}
\setlength{\bibhang}{1em}
\newlength{\bibsep}
 {\@listi \global\bibsep\itemsep \global\advance\bibsep by\parsep}
\newenvironment{bibsection}%
        {\vspace{-\baselineskip}\begin{list}{}{%
       \setlength{\leftmargin}{\bibhang}%
       \setlength{\itemindent}{-\leftmargin}%
       \setlength{\itemsep}{\bibsep}%
       \setlength{\parsep}{\z@}%
        \setlength{\partopsep}{0pt}%
        \setlength{\topsep}{0pt}}}
        {\end{list}\vspace{-.6\baselineskip}}
\makeatother

% Layout: Puts the section titles on left side of page
\reversemarginpar

%
%         PAPER SIZE, PAGE NUMBER, AND DOCUMENT LAYOUT NOTES:
%
% The next \usepackage line changes the layout for CV style section
% headings as marginal notes. It also sets up the paper size as either
% letter or A4. By default, letter was used. If A4 paper is desired,
% comment out the letterpaper lines and uncomment the a4paper lines.
%
% As you can see, the margin widths and section title widths can be
% easily adjusted.
%
% ALSO: Notice that the includefoot option can be commented OUT in order
% to put the PAGE NUMBER *IN* the bottom margin. This will make the
% effective text area larger.
%
% IF YOU WISH TO REMOVE THE ``of LASTPAGE'' next to each page number,
% see the note about the +LP and -LP lines below. Comment out the +LP
% and uncomment the -LP.
%
% IF YOU WISH TO REMOVE PAGE NUMBERS, be sure that the includefoot line
% is uncommented and ALSO uncomment the \pagestyle{empty} a few lines
% below.
%

%% Use these lines for letter-sized paper
%\usepackage[paper=letterpaper,
%            %includefoot, % Uncomment to put page number above margin
%            marginparwidth=1.2in,     % Length of section titles
%           marginparsep=.05in,       % Space between titles and text
%            margin=1in,               % 1 inch margins
%            includemp]{geometry}

% Use these lines for A4-sized paper
\usepackage[paper=a4paper,
            %includefoot, % Uncomment to put page number above margin
            marginparwidth=25mm,    % Length of section titles
            marginparsep=5mm,       % Space between titles and text
            margin=20mm,              % 25mm margins
            includemp]{geometry}

%% More layout: Get rid of indenting throughout entire document
\setlength{\parindent}{0in}

%% This gives us fun enumeration environments. compactitem will be nice.
\usepackage{paralist}

%% Reference the last page in the page number
%
% NOTE: comment the +LP line and uncomment the -LP line to have page
%       numbers without the ``of ##'' last page reference)
%
% NOTE: uncomment the \pagestyle{empty} line to get rid of all page
%       numbers (make sure includefoot is commented out above)
%
\usepackage{fancyhdr,lastpage}
\pagestyle{fancy}
%\pagestyle{empty}      % Uncomment this to get rid of page numbers
\fancyhf{}\renewcommand{\headrulewidth}{0pt}
\fancyfootoffset{\marginparsep+\marginparwidth}
\newlength{\footpageshift}
\setlength{\footpageshift}
          {0.5\textwidth+0.5\marginparsep+0.5\marginparwidth-2in}
\lfoot{\hspace{\footpageshift}%
       \parbox{4in}{\, \hfill %
                    \arabic{page} of \protect\pageref*{LastPage} % +LP
%                    \arabic{page}                               % -LP
                    \hfill \,}}

% Finally, give us PDF bookmarks
\usepackage{color,hyperref}
\definecolor{darkblue}{rgb}{0.0,0.0,0.3}
\hypersetup{colorlinks,breaklinks,
            linkcolor=darkblue,urlcolor=darkblue,
            anchorcolor=darkblue,citecolor=darkblue}

%%%%%%%%%%%%%%%%%%%%%%%% End Document Setup %%%%%%%%%%%%%%%%%%%%%%%%%%%%


%%%%%%%%%%%%%%%%%%%%%%%%%%% Helper Commands %%%%%%%%%%%%%%%%%%%%%%%%%%%%

% The title (name) with a horizontal rule under it
% (optional argument typesets an object right-justified across from name
%  as well)
%
% Usage: \makeheading{name}
%        OR
%        \makeheading[right_object]{name}
%
% Place at top of document. It should be the first thing.
% If ``right_object'' is provided in the square-braced optional
% argument, it will be right justified on the same line as ``name'' at
% the top of the CV. For example:
%
%       \makeheading[\emph{Curriculum vitae}]{Your Name}
%
% will put an emphasized ``Curriculum vitae'' at the top of the document
% as a title. Likewise, a picture could be included:
%
%   \makeheading[\includegraphics[height=1.5in]{my_picutre}]{Your Name}
%
% the picture will be flush right across from the name.
\newcommand{\makeheading}[2][]%
        {\hspace*{-\marginparsep minus \marginparwidth}%
         \begin{minipage}[t]{\textwidth+\marginparwidth+\marginparsep}%
             {\large \bfseries #2 \hfill #1}\\[-0.15\baselineskip]%
                 \rule{\columnwidth}{1pt}%
         \end{minipage}}

% The section headings
%
% Usage: \section{section name}
%
% Follow this section IMMEDIATELY with the first line of the section
% text. Do not put whitespace in between. That is, do this:
%
%       \section{My Information}
%       Here is my information.
%
% and NOT this:
%
%       \section{My Information}
%
%       Here is my information.
%
% Otherwise the top of the section header will not line up with the top
% of the section. Of course, using a single comment character (%) on
% empty lines allows for the function of the first example with the
% readability of the second example.
\renewcommand{\section}[2]%
        {\pagebreak[3]\vspace{1.3\baselineskip}%
         \phantomsection\addcontentsline{toc}{section}{#1}%
         \hspace{0in}%
         \marginpar{
         \raggedright \scshape #1}#2}

% An itemize-style list with lots of space between items
\newenvironment{outerlist}[1][\enskip\textbullet]%
        {\begin{itemize}[#1]}{\end{itemize}%
         \vspace{-.6\baselineskip}}

% An environment IDENTICAL to outerlist that has better pre-list spacing
% when used as the first thing in a \section
\newenvironment{lonelist}[1][\enskip\textbullet]%
        {\vspace{-\baselineskip}\begin{list}{#1}{%
        \setlength{\partopsep}{0pt}%
        \setlength{\topsep}{0pt}}}
        {\end{list}\vspace{-.6\baselineskip}}

% An itemize-style list with little space between items
\newenvironment{innerlist}[1][\enskip\textbullet]%
        {\begin{compactitem}[#1]}{\end{compactitem}}

% An environment IDENTICAL to innerlist that has better pre-list spacing
% when used as the first thing in a \section
\newenvironment{loneinnerlist}[1][\enskip\textbullet]%
        {\vspace{-\baselineskip}\begin{compactitem}[#1]}
        {\end{compactitem}\vspace{-.6\baselineskip}}

% To add some paragraph space between lines.
% This also tells LaTeX to preferably break a page on one of these gaps
% if there is a needed pagebreak nearby.
\newcommand{\blankline}{\quad\pagebreak[3]}
\newcommand{\halfblankline}{\quad\vspace{-0.5\baselineskip}\pagebreak[3]}

% Uses hyperref to link DOI
\newcommand\doilink[1]{\href{http://dx.doi.org/#1}{#1}}
\newcommand\doi[1]{doi:\doilink{#1}}

% For \url{SOME_URL}, links SOME_URL to the url SOME_URL
\providecommand*\url[1]{\href{#1}{#1}}
% Same as above, but pretty-prints SOME_URL in teletype fixed-width font
\renewcommand*\url[1]{\href{#1}{\texttt{#1}}}

% For \email{ADDRESS}, links ADDRESS to the url mailto:ADDRESS
\providecommand*\email[1]{\href{mailto:#1}{#1}}
% Same as above, but pretty-prints ADDRESS in teletype fixed-width font
%\renewcommand*\email[1]{\href{mailto:#1}{\texttt{#1}}}

%\providecommand\BibTeX{{\rm B\kern-.05em{\sc i\kern-.025em b}\kern-.08em
%    T\kern-.1667em\lower.7ex\hbox{E}\kern-.125emX}}
%\providecommand\BibTeX{{\rm B\kern-.05em{\sc i\kern-.025em b}\kern-.08em
%    \TeX}}
\providecommand\BibTeX{{B\kern-.05em{\sc i\kern-.025em b}\kern-.08em
    \TeX}}
\providecommand\Matlab{\textsc{Matlab}}

%%%%%%%%%%%%%%%%%%%%%%%% End Helper Commands %%%%%%%%%%%%%%%%%%%%%%%%%%%

%%%%%%%%%%%%%%%%%%%%%%%%% Begin CV Document %%%%%%%%%%%%%%%%%%%%%%%%%%%%

\begin{document}
\makeheading{Jiajie Peng}

\section{Contact Information}
%
% NOTE: Mind where the & separators and \\ breaks are in the following
%       table.
%
% ALSO: \rcollength is the width of the right column of the table
%       (adjust it to your liking; default is 1.85in).
%
%\newlength{\rcollength}\setlength{\rcollength}{2.05in}%
%
\begin{tabular}[t]{ll}
School of Computer Science and Technolodgy& \textit{Voice:} +86-187 4513 7124 \\
315 Zonghe Building, 92 West Dazhi Street , & \textit{E-mail:} \email{jiajiepeng@hit.edu.cn}\\
Nan Gang District, Harbin, China  & %\textit{Homepage:} \href{http://www.hlt.utdallas.edu/~zywei}{http://www.hlt.utdallas.edu/$\scriptsize{\sim}$zywei}\\
\end{tabular}

%\section{Objective}
%
%Placement in an academic position (i.e., faculty, postdoctoral, or
%research scientist) that allows for advanced research in distributed
%complex adaptive systems (i.e., modeling, analysis, design, and
%verification) with a particular focus on the control of engineered
%agents (e.g., for communications, control, software, and electronics
%applications) and the analysis of biological phenomena (e.g.,
%self-organization, ecological rationality)
%
\section{Research Interests}
%
Bioinformatics, data mining, data-driven ontology construction, gene ontology based semantic similarity, biological network reconstruction 

%\section{Current Position}
%\textbf{University of Texas at Dallas (UTD)}, Richardson, TX, USA \hfill Dec. 2014 - Present \\
%Postdoctoral Associate, Human Language Technology Research Institute\\
%Mentor: Yang Liu

\section{Education}
\textbf{Harbin Institute of Technolodgy}, Harbin, China \hfill Aug. 2010 - now \\
Ph.D., Computer Science and Technolodgy\\
Advisor: Yadong Wang\\
\\
\textbf{Michigan State University}, East Lansing, MI, USA\\
Visiting Scholar, MSU-DOE Plant Research Laboratory \hfill Sep. 2010 - Sep. 2012\\
Mentor: Jin Chen\\

\textbf{Harbin Institute of Technology}, Harbin, China\\
M.Phil., Computer Science and Technology \hfill Aug. 2008 - Jul. 2010\\
B.Sc., Computer Science and Technology \hfill Aug. 2004 - Jul. 2008\\


\section{Publications}
\textbf{- Conference Papers}
\begin{itemize}
	\item \textbf{Zhongyu Wei} and Wei Gao, \emph{Gibberish, Assistant, or Master? Using Tweets Linking to News for Extractive Single-Document Summarization}, ACM SigIR 2015, accepted.
	\item Gaoyan Ou, Wei Chen, Dongqing Yang, \textbf{Zhongyu Wei}, Binyang Li, Tengjiao Wang and Kam-Fai Wong, \emph{Exploiting Community Emotion for Microblog Event Detection},  EMNLP 2014.
	\item \textbf{Zhongyu Wei}, Wei Gao, \emph{Utilizing Microblogs for Automatic News Highlights Extraction}, COLING 2014.
	 \item Binyang Li, Lanjun Zhou, \textbf{Zhongyu Wei}, Kam-Fai Wong, Ruifeng Xu and Yunqing Xia, \emph{Web Information Mining and Decision Support Platform for the Modern Service Industry}, ACL 2014, demo paper.
	\item Lanjun Zhou, Binyang Li, \textbf{Zhongyu Wei} and Kam-Fai Wong, \emph{An Open Discourse Corpus for Chinese with Annotated Explicit Discourse Connectives}, LREC 2014.
	 \item \textbf{Zhongyu Wei}, Junwen Chen, Wei Gao, Binyang Li, Lanjun Zhou, Yulan He and Kam-Fai Wong, \emph{An Empirical Study on Uncertainty Identification in Social Media Context}, ACL 2013.
    \item \textbf{Zhongyu Wei}, Yulan He, Wei Gao, Lanjun Zhou, Binyang Li and Kam-Fai Wong, \emph{Mainstream Media Behavior Analysis on Twitter: A Case Study on UK General Election}, ACM HyperText 2013.
    \item Rebecca Ferguson, \textbf{Zhongyu Wei}, Yulan He, Simon Buckingham Shum, \emph{An Evaluation of Learning Analytics to Identify Exploratory Dialogue in Online Discussions},  ACM LAK 2013.
     \item  Lanjun Zhou, Wei Gao, Binyang Li, \textbf{Zhongyu Wei}, and Kam-Fai Wong. \emph{Cross-lingual Identification of Ambiguous Discourse Connectives for Resource-Poor Language}, COLING 2012.
      \item  Yulan He, Hassan Saif, \textbf{Zhongyu Wei}, and Kam-Fai Wong, \emph{Qunatising Opinions for Political Tweets Analysis}, LREC 2012.
      \item  Binyang Li, Lanjun Zhou, Wei Gao, Kam-Fai Wong and \textbf{Zhongyu Wei}, \emph{An Effective Approach for Topic-Specific Opinion Summarization}, AIRS 2011.
    \item  Lanjun Zhou, Binyang Li, Wei Gao, \textbf{Zhongyu Wei} and Kam-Fai Wong, \emph{Unsupervised Discovery of Discourse Relations for Eliminating Intra-sentence Polarity Ambiguities}, EMNLP 2011.
    \item  Jun Xu, Qingcai Chen, Xiaolong Wang and \textbf{Zhongyu Wei}, \emph{One-class classification models for financial industry information recommendation}, ICMLC 2010.
\end{itemize}

\textbf{- Workshop Papers and Technical Reports}
\begin{itemize}
     \item Walid Magdy, Wei Gao, Tarek El-Ganainy and  \textbf{Zhongyu Wei}, \emph{QCRI at TREC 2014: Applying the KISS Principle for the TTG Task in the Microblog Track}, Microblog Track at TREC2014. (2nd place among 13 groups)
    \item \textbf{Zhongyu Wei}, Wei Gao, \emph{Ranking Model Selection and Fusion for Effective Microblog Search}, SIGIR 2014 Workshop on Social Media Retrieval and Analysis(SoMeRA), Gold Coast, Australia, July, 2014.
    \item Tarek El-Ganainy, \textbf{Zhongyu Wei}, Walid Magdy, and Wei Gao, \emph{QCRI at TREC 2013 Microblog Track}, Microblog Track at TREC2013. (2nd place among 65 automatic runs)
    \item \sloppypar{Wei Gao, \textbf{Zhongyu Wei} and Kam-Fai Wong, \emph{Microblog Search and Filtering with Time Sensitive Feedback and Thresholding Based on BM25}, Microblog Track at TREC2012.}
     \item  \textbf{Zhongyu Wei}, Wei Gao, Lanjun Zhou, Binyang Li, and Kam-Fai Wong, \emph{Exploring Tweets Normalization and Query Time Sensitivity for Twitter Search}, Microblog Track at TREC2011.
\end{itemize}

\textbf{- Journal Articles}
\begin{itemize}
       \item Binyang Li, Lanjun Zhou, Kam-Fai Wong, and \textbf{Zhongyu Wei}. \emph{An Effective Information Representation for Opinion Retrieval}, Journal of the American Society for Information Science and Technology (JASIST), 2015, accepted for publication.
       \item \textbf{Zhongyu Wei}, Yulan He, Simon Buckingham Shum, Rebecca Ferguson, Wei Gao and Kam-Fai Wong, \emph{A Self-Training Framework for Automatic Identification of Exploratory Dialogue}, International Journal of Computational Linguistics and Applications (IJCLA), 4(1): 111 - 126 (2013).
        \item Binyang Li, Kam-Fai Wong, Lanjun Zhou, \textbf{Zhongyu Wei}, and Jun Xu, \emph{Pests Hidden in Your Fans: An Effective Approach for Opinion Leader Discovery}, In Chinese Computational Linguistics and Natural Language Processing Based on Naturally Annotated Big Data, 227-237 (2013).
        \item \textbf{Zhongyu Wei}, Jun Xu, Xiaolong Wang. \emph{One-class Classification based Finance News Story Recommendation}, Journal of Computational Information Systems (JCIS), 5(6): 1625-1631 (2009).
 \end{itemize}

\textbf{- Under Review}
\begin{itemize}
     \item \textbf{Zhongyu Wei}, Yang Liu, Chen Li, \emph{An Empirical Study on Using Topic Structure for News Highlights Generation}.
     \item \textbf{Zhongyu Wei}, Yang Liu, Chen Li, \emph{Using Tweets to Help Sentence Compression for News Highlights Generation}.
    \item \textbf{Zhongyu Wei}, Wei Gao, Kam-Fai Wong, Yang Liu, \emph{An Unsupervised Method of Using Tweets for News Highlights Generation}.
     \item Jing Li, Wei Gao, \textbf{Zhongyu Wei}, Kam-Fai Wong, \emph{Summarizing Microblog Repost Trees with Structure Information}.
     \item Jing Ma, Wei Gao, \textbf{Zhongyu Wei}, Kam-Fai Wong, \emph{Detect Rumors Using Time Series of Social Context Information on Microblogging Websites}.
\end{itemize}

\section{Research}
%\textbf{University of Texas at Dallas}\\
%\emph{Postdoctoral Associate} \hfill Dec. 2014 - Present\\
\textbf{Chinese University of Hong Kong}\\
\emph{Research Assistant} \hfill Aug. 2010 - Dec. 2014\\

1. Utilizing Microblogs for Automatic News Highlights Generation\\
    \emph{Collaboration project with Qatar Computing Institute of Technology}
\begin{itemize}
\itemsep-0.2em
	\item \underline{OVERVIEW}: This project aims to use relevant tweets of a news article to help generate better news summary. We proposed three strategies: (1) extract news sentence with the help of relevant tweets; (2) extract tweets as news highlights; (3) use relevant tweets to help select and compress the news sentence as highlights. 
	\item \underline{CORPUS}: We constructed two corpora for evaluation, including CNN news articles and their relevant tweets. One corpus includes human generated summary collected from Amazon Mechanical Turk. 
	\item \underline{PUBLICATION}: COLING2014, SIGIR2015, two papers under review
\end{itemize}

2. Emergent Rumor Detection and Credibility Ranking
\begin{itemize}
\itemsep-0.2em
	\item \underline{OVERVIEW}: This project aims to identify rumor events from social media and generate tweets summary for each identified rumor event. We proposed a rumor detection framework considering three factors: (1) text uncertainty, (2) temporal characteristics (3) opinion controversy.
	\item \underline{CORPUS}: We constructed one corpus for rumor detection, including events consisting of related microblogs from a Chinese microblogging website. Human annotations were collected from Amazon Mechanical Turk.
	\item \underline{PUBLICATION}: ACL2013, two papers under review
\end{itemize}

3. Web Information Mining and Decision Support Platform for the Modern Service Industry
\begin{itemize}
\itemsep-0.2em
	\item \underline{OVERVIEW}: This project aims to provide enterprises with the services of retrieving news from websites, extracting commercial information, exploring customers' opinions, and analyzing collaborative/competitive social networks. The core technologies have been applied to the pillar industries of Hong Kong, including innovative finance, modem logistics, information technology, etc. 
	\item \underline{SYSTEM}: \href{http://sepc111.se.cuhk.edu.hk:8080/modest}{[MODEST]} (in Chinese)
	\item \underline{PUBLICATION}: ACL2014
\end{itemize}

4. Microblog Search
\begin{itemize}
\itemsep-0.2em
	\item \underline{OVERVIEW}: We proposed to improve microblog search performance from two aspects: (1) utilize external information for pseudo relevance feedback (PRF) ; (2) re-rank search results by combing several state-of-the-art ranking algorithms.  
	\item \sloppypar{\underline{PUBLICATION}: Technical reports on TREC2011, TREC2012, TREC2013, TREC2014 and SoMeRA2015}
	\item \underline{IMPACT}: We participated in microblog track in TREC and won second place for year 2013 and 2014. 
\end{itemize}

5. Social Media Analysis
\begin{itemize}
\itemsep-0.2em
	\item \underline{OVERVIEW}: This project aims to investigate the behavior of mainstream media on Twitter and study how they exert their influence to shape public opinion during the UK's 2010 General Election.
	\item \underline{PUBLICATION}: ACM HyperText2013
\end{itemize}

\vspace{10pt}
\textbf{Knowledge Media Institute}\\
\emph{Research Associate} \hfill Apr. 2012 - Jun. 2012\\

1. Exploratory Discourse Detection
\begin{itemize}
\itemsep-0.2em
	\item \underline{OVERVIEW}: This project aims to identify exploratory dialogues from online learning material. We proposed a self-training framework to identify exploratory dialogue.
	\item \underline{CORPUS}: We constructed a corpus including online meeting text with human annotations collected from two education experts. The annotation guideline was generated based on collaboration with annotators. 
	\item \underline{SYSTEM}: The technique has been applied to \href{http://sociallearn.open.ac.uk/public}{Social Learn Platform}.
	\item \underline{PUBLICATION}: IJCLA, ACM LAK2013
\end{itemize}

\section{Conference Review}
\textbf{Secondary Reviewer}\\\\
2015: NAACL, ACL\\
2014: SigIR, COLING, EMNLP\\
2013: ACL, EMNLP, IJCAI, RANLP, AIRS\\
2012: ACL, EMNLP, PACLIC, AIRS\\
2011: EMNLP

\section{Professional Skills}
\textbf{General}: Machine learning for NLP, fast modeling and prototyping, data analysis, familiar with human annotation collection for NLP task%, familiar with Twitter data processing

\textbf{Programming}: Java, Python, Unix/Linux shell scripting, R, C/C++


\section{Teaching Experience}
\textbf{Teaching Assistant} at Systems Engineering and Engineering Management, CUHK\\

Undergraduate course "Engineering Entrepreneurship" \hfill Spring 2011, 2012, 2013, 2014\\
Undergraduate course "Information Systems Analysis and Design" \hfill Fall 2012\\
Undergraduate course "Information Technology Management" \hfill Fall 2012, 2013\\
Undergraduate course "Information Systems Management" \hfill Fall 2011\\

\section{Mentorship}
Jing Ma, Research Assistant, Chinese University of Hong Kong \hfill Spring 2015 - Present\\
Jing Li, PhD, Chinese University of Hong Kong \hfill Spring 2014 - Present\\
Jun Wen, PhD, Chinese University of Hong Kong \hfill Spring 2014 - Present\\


\section{Honors and Awards}
Excellent Graduate in Heilongjiang Province (top 1\%) \hfill 2010\\
Excellent Postgraduate Student in HIT (top 1\%) \hfill 2010\\
Excellent Graduate in Heilongjiang Province (top 1\%) \hfill 2008 \\
Excellent Student Leader in Heilongjiang Province (top 1\%) \hfill 2006\\

\section{Referees}

\begin{tabular}{ll}
\href{http://www.cintec.cuhk.edu.hk/kfwong/}{\textbf{Kam-Fai Wong}} & \href{http://www.qcri.com/our-people/bio?pid=44&par=acc&name=WeiGao}{\textbf{Wei Gao}}\\
PhD, Professor & PhD, Scientist\\
The Chinese University of Hong Kong & Qatar Computing Research Institute\\
Email: kfwong@se.cuhk.edu.hk & Email: wgao@qf.org.qa\\
\\
%\href{http://www1.aston.ac.uk/eas/staff/dr-yulan-he/} {\textbf{Dr. Yulan HE}} & \href{http://cs.hitsz.edu.cn/teachers/t1/1181827169.html}{\textbf{Professor Xiaolong} WANG}\\
%PhD, Senior lecturer & PhD, Professor\\
%Aston University & Harbin Institute of Technology\\
%Email: y.he9@aston.ac.uk & Email: wangxl@insun.hit.edu.cn\\
%\end{tabular}

%\href{http://www.cintec.cuhk.edu.hk/kfwong/}{\textbf{Professor Kam-Fai WONG}} & \href{http://www1.se.cuhk.edu.hk/~hcheng/}{\textbf{Prof. Hong CHENG}}\\
%PhD, Professor & PhD, Assistant Professor\\
%The Chinese University of Hong Kong & The Chinese University of Hong Kong\\
%Email: kfwong@se.cuhk.edu.hk & Email: hcheng@se.cuhk.edu.hk\\
%\\
\href{http://www1.aston.ac.uk/eas/staff/dr-yulan-he/} {\textbf{Yulan He}} & \\
PhD, Senior lecturer & \\
Aston University & \\
Email: y.he9@aston.ac.uk &\\

%\href{http://www1.aston.ac.uk/eas/staff/dr-yulan-he/} {\textbf{Yulan He}} &\href{http://www.hlt.utdallas.edu/~yangl}{\textbf{Yang Liu}} \\
%PhD, Senior lecturer & PhD, Associate Processor \\
%Aston University & The University of Texas at Dallas\\
%Email: y.he9@aston.ac.uk &Email: yangl@hlt.utdallas.edu\\
\end{tabular}

\end{document} 